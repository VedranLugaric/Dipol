\chapter{Arhitektura i dizajn sustava}
		
		 Arhitektura sustava ima hijerarhijsku strukturu u kojoj svaki sloj komunicira isključivo s neposredno susjednim slojevima. Naš sustav sastoji se od pet glavnih slojeva: Korisničko sučelje, Kontroler, Servis, Repozitorij i Baza podataka. Korisničko sučelje, ili User Interface (UI) omogućava interakciju između korisnika i računala. Korisničko sučelje našeg sustava razvijeno je uz pomoć Reacta, JavaScript biblioteke koja olakšava stvaranje korisničkih sučelja. Korisničko sučelje šalje zahtjeve kontroleru temeljem korisničkih akcija i koristi JSON (JavaScript Object Notation) datoteke za prijenos podataka. Kontroler, koristeći REST API (Representational State Transfer), upravlja zahtjevima vanjskih korisnika i odgovara na njih. U većini slučajeva, kontroler radi s podacima u JSON formatu. Servis je odgovoran za upravljanje i obradu podataka dobivenih od korisničkog sučelja putem kontrolera i baze podataka putem repozitorija. Repozitorij se koristi za komunikaciju s bazom podataka i sadrži funkcionalnosti za pronalaženje određenih objekata iz baze.	

		Arhitektura se može podijeliti na tri osnovna podsustava: Web poslužitelj, Web aplikacija i Baza podataka. Web preglednik omogućuje korisnicima pregledavanje web-stranica i pristup multimedijalnom sadržaju na internetu. Svaki web preglednik djeluje kao prevoditelj, interpretirajući web-stranice napisane u kodu i prikazujući ih korisnicima na razumljiv način. Korisnici šalju zahtjeve web poslužitelju putem web preglednika, a web poslužitelj igra ključnu ulogu u radu web aplikacije. Njegova primarna zadaća je omogućiti komunikaciju između korisnika i aplikacije putem HTTP (HyperText Transfer Protocol) protokola, standardnog načina prijenosa informacija na webu. Web poslužitelj pokreće web aplikaciju i proslijeđuje joj korisničke zahtjeve.

		Korisnici koriste web aplikaciju za obradu svojih zahtjeva. Web aplikacija obrađuje te zahtjeve i, ovisno o njihovoj prirodi, pristupa bazi podataka putem repozitorija. Nakon obrade zahtjeva, web aplikacija preko web poslužitelja vraća odgovor u obliku HTML dokumenta koji korisnici vide u svom web pregledniku. Za razvoj web aplikacije koristi se programski jezik Python zajedno s .NET radnim okvirom i JavaScriptom, a razvojno okruženje je Microsoft Visual Studio. Arhitektura sustava temelji se na konceptu Model-View-Controller (MVC), arhitekturnog obrasca koji se često koristi u razvoju softverskih aplikacija kako bi se postigla jasna organizacija i odvojenost različitih dijelova aplikacije. Sastoji se od tri osnovne komponente:
	\begin{itemize}
		\item 	{Model: predstavlja središnju komponentu sustava. On je odgovoran za upravljanje podacima logikom i pravilima aplikacije. Neovisan je o korisničkom sučelju i često sadrži dinamičke podatkovne strukture koje predstavljaju stanje aplikacije. Kada se dogodi promjena u podacima ili stanju aplikacije, Model obavještava ostale komponente sustava o tim promjenama.}
		\item 	{View: komponenta odgovorna za prikaz podataka korisnicima. To uključuje sve vizualne elemente sučelja, kao što su grafovi, tablice, forme i slično. View omogućava korisnicima da vide i koriste podatke iz Modela na način koji im je razumljiv.}
		\item 	{Controller: komponenta koja prima ulazne podatke od korisnika ili drugih izvora i upravlja njima. Kontrolira korisničke zahtjeve i daljnju interakciju s Modelom i View-om. Kada korisnik izvrši neku radnju, Controller reagira na tu akciju i donosi odluke o tome kako će se to odraziti na Model i kako će se ažurirati View.}
	\end{itemize}

	
		

		

				
		\section{Baza podataka}
					
		\textit{Potrebno je opisati koju vrstu i implementaciju baze podataka ste odabrali, glavne komponente od kojih se sastoji i slično.}
		
			\subsection{Opis tablica}
			

				\textit{Svaku tablicu je potrebno opisati po zadanom predlošku. Lijevo se nalazi točno ime varijable u bazi podataka, u sredini se nalazi tip podataka, a desno se nalazi opis varijable. Svjetlozelenom bojom označite primarni ključ. Svjetlo plavom označite strani ključ}
				
				
				\begin{longtblr}[
					label=none,
					entry=none
					]{
						width = \textwidth,
						colspec={|X[6,l]|X[6, l]|X[20, l]|}, 
						rowhead = 1,
					} %definicija širine tablice, širine stupaca, poravnanje i broja redaka naslova tablice
					\hline \SetCell[c=3]{c}{\textbf{korisnik - ime tablice}}	 \\ \hline[3pt]
					\SetCell{LightGreen}IDKorisnik & INT	&  	Lorem ipsum dolor sit amet, consectetur adipiscing elit, sed do eiusmod  	\\ \hline
					korisnickoIme	& VARCHAR &   	\\ \hline 
					email & VARCHAR &   \\ \hline 
					ime & VARCHAR	&  		\\ \hline 
					\SetCell{LightBlue} primjer	& VARCHAR &   	\\ \hline 
				\end{longtblr}
				
				
			
			\subsection{Dijagram baze podataka}
				\textit{ U ovom potpoglavlju potrebno je umetnuti dijagram baze podataka. Primarni i strani ključevi moraju biti označeni, a tablice povezane. Bazu podataka je potrebno normalizirati. Podsjetite se kolegija "Baze podataka".}
			
			\eject
			
			
		\section{Dijagram razreda}
		
			\textit{Potrebno je priložiti dijagram razreda s pripadajućim opisom. Zbog preglednosti je moguće dijagram razlomiti na više njih, ali moraju biti grupirani prema sličnim razinama apstrakcije i srodnim funkcionalnostima.}\\
			
			\textbf{\textit{dio 1. revizije}}\\
			
			\textit{Prilikom prve predaje projekta, potrebno je priložiti potpuno razrađen dijagram razreda vezan uz \textbf{generičku funkcionalnost} sustava. Ostale funkcionalnosti trebaju biti idejno razrađene u dijagramu sa sljedećim komponentama: nazivi razreda, nazivi metoda i vrste pristupa metodama (npr. javni, zaštićeni), nazivi atributa razreda, veze i odnosi između razreda.}\\
			
			\textbf{\textit{dio 2. revizije}}\\			
			
			\textit{Prilikom druge predaje projekta dijagram razreda i opisi moraju odgovarati stvarnom stanju implementacije}
			
			
			
			\eject
		
		\section{Dijagram stanja}
			
			
			\textbf{\textit{dio 2. revizije}}\\
			
			\textit{Potrebno je priložiti dijagram stanja i opisati ga. Dovoljan je jedan dijagram stanja koji prikazuje \textbf{značajan dio funkcionalnosti} sustava. Na primjer, stanja korisničkog sučelja i tijek korištenja neke ključne funkcionalnosti jesu značajan dio sustava, a registracija i prijava nisu. }
			
			
			\eject 
		
		\section{Dijagram aktivnosti}
			
			\textbf{\textit{dio 2. revizije}}\\
			
			 \textit{Potrebno je priložiti dijagram aktivnosti s pripadajućim opisom. Dijagram aktivnosti treba prikazivati značajan dio sustava.}
			
			\eject
		\section{Dijagram komponenti}
		
			\textbf{\textit{dio 2. revizije}}\\
		
			 \textit{Potrebno je priložiti dijagram komponenti s pripadajućim opisom. Dijagram komponenti treba prikazivati strukturu cijele aplikacije.}